\newpage
\section{Kết Luận}

Việc khớp đặc trưng giữa các hình ảnh là một bài toán quan trọng trong lĩnh vực xử lý ảnh số và thị giác máy tính. Trong bài tiểu luận này, nhóm nghiên cứu đã tìm hiểu và áp dụng các kỹ thuật khớp đặc trưng như SIFT, ORB, FLANN, cùng với các kỹ thuật lọc như Lowe’s Ratio Test và RANSAC để nâng cao độ chính xác của các điểm khớp. Nội dung nghiên cứu đã tổng hợp các nguyên lý lý thuyết và triển khai qua thực nghiệm, từ đó mang lại những kết quả đáng khích lệ. Các phương pháp này không chỉ giúp cải thiện hiệu quả xử lý mà còn cung cấp những công cụ mạnh mẽ để giải quyết các vấn đề thực tế liên quan đến so khớp hình ảnh.\\

Trong quá trình triển khai, các thuật toán phát hiện đặc trưng như SIFT đã chứng minh được độ chính xác và độ ổn định cao, đặc biệt trong việc xử lý những hình ảnh có nhiều chi tiết phức tạp. SURF với tốc độ xử lý nhanh hơn SIFT cũng là một lựa chọn hiệu quả trong các bài toán yêu cầu thời gian xử lý ngắn hơn, nhưng độ chính xác có thể giảm trong trường hợp chi tiết phức tạp hoặc biến đổi hình ảnh lớn. ORB, là một thuật toán nhanh và tiết kiệm tài nguyên, phù hợp với các ứng dụng thời gian thực hoặc các hệ thống có hạn chế về tài nguyên, mặc dù độ chính xác của ORB thường thấp hơn so với SIFT và SURF trong các bài toán yêu cầu độ chi tiết cao.\\

FLANN với khả năng khớp nhanh chóng đã giúp đơn giản hóa quá trình tìm đối tượng khớp trong tập dữ liệu lớn. Để nâng cao độ chính xác, Lowe's Ratio Test đã loại bỏ nhiều kết quả sai, trong khi RANSAC giúp lọc những khớp đặc trưng bị ảnh hưởng bởi nhiễu tạp chất hoặc đối tượng ngoại lai. Các kết quả hình ảnh thực nghiệm đã minh chứng cho sự hợp lý của những kỹ thuật này. Thêm vào đó, việc kết hợp các phương pháp này trong một quy trình thống nhất đã tạo ra một hệ thống mạnh mẽ, hiệu quả và đáng tin cậy trong việc giải quyết bài toán khớp đặc trưng.\\

Tuy nhiên, trong quá trình nghiên cứu, một số hạn chế cũng được ghi nhận. Các thuật toán như SIFT dù có độ chính xác cao nhưng thời gian xử lý lại tương đối lâu, đặc biệt khi làm việc với tập dữ liệu lớn. SURF, dù nhanh hơn, vẫn gặp khó khăn trong các trường hợp hình ảnh bị biến dạng lớn hoặc góc chụp khác biệt (dựa theo bài báo so sánh). ORB tuy nhanh nhưng thường không đủ chính xác trong các bài toán phức tạp. FLANN có độ nhanh nhưng kết quả có thể bị sai khi dữ liệu đặc trưng không đủ phong phú. Hơn nữa, đối với các hình ảnh bị biến dạng quá lớn hoặc góc chụp quá khác biệt, kết quả không đạt được mức mong muốn. Những hạn chế này chỉ ra rằng việc kết hợp thêm các phương pháp mới hoặc cải tiến các thuật toán hiện tại là cần thiết để đạt được hiệu suất cao hơn.\\

Hướng nghiên cứu trong tương lai có thể tập trung vào việc áp dụng các thuật toán tiên tiến hơn như SuperPoint hay các mô hình học sâu cho việc khớp đặc trưng. Việc tối ưu hóa hiệu suất thông qua tối ưu phần cứng và đánh giá kỹ lưỡng trải nghiệm cũng là những hướng đi rõ rệt. Hơn nữa, tích hợp các thuật toán này vào những hệ thống thực tế như nhận diện khuôn mặt hay tính năng định vị đồng thời là bước nhảy vọt quan trọng. Các nghiên cứu này không chỉ cải thiện độ chính xác mà còn mở rộng khả năng ứng dụng của các kỹ thuật khớp đặc trưng vào nhiều lĩnh vực mới, đặc biệt trong các ngành công nghiệp sử dụng công nghệ thị giác máy tính.\\

Tổng quát lại, nghiên cứu về việc khớp đặc trưng giữa các hình ảnh đã minh chứng cho tầm quan trọng và tiềm năng lớn của lĩnh vực này trong xử lý ảnh và thị giác máy tính. Dù còn những hạn chế, đòi hỏi các nghiên cứu chuyên sâu để tối ưu cũng như tìm ra giải pháp tốt hơn, nhưng những kết quả đạt được là bước đệm quan trọng trong việc phát triển các ứng dụng thực tế. Sự kết hợp giữa lý thuyết và thực nghiệm trong nghiên cứu này đã tạo ra một nền tảng vững chắc cho nhóm nghiên cứu phát triển công trình tiếp theo trong tương lai.\\

Toàn bộ mã nguồn triển khai của các phương pháp được thảo luận trong bài viết có thể tham khảo tại kho lưu trữ GitHub của nhóm nghiên cứu: \url{https://github.com/hienlq16103/digital-image-processing-and-computer-vision}.