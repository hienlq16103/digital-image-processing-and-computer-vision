\documentclass[a4paper]{article}
\usepackage{a4wide,amssymb,epsfig,latexsym,multicol,array,hhline,fancyhdr}
\usepackage{vntex}
\usepackage{amsmath}
\usepackage{cases}
\usepackage{lastpage}
\usepackage[lined,boxed,commentsnumbered]{algorithm2e}
\usepackage{enumerate}
\usepackage{color}
\usepackage{graphicx}							% Standard graphics package
\usepackage{array}
\usepackage{tabularx, caption}
\usepackage{multirow}
\usepackage{multicol}
\usepackage{rotating}
\usepackage{graphics}
\usepackage{geometry}
\usepackage{setspace}
\usepackage{epsfig}
\usepackage{tikz}
\usepackage{amsfonts}
\usetikzlibrary{arrows,snakes,backgrounds}
\usepackage{hyperref}
%\usepackage{txfonts}
\usepackage{mathdots}



\hypersetup{urlcolor=blue,linkcolor=black,citecolor=black,colorlinks=true} 
%\usepackage{pstcol} 								% PSTricks with the standard color package

\newtheorem{theorem}{{\bf Theorem}}
\newtheorem{property}{{\bf Property}}
\newtheorem{proposition}{{\bf Proposition}}
\newtheorem{corollary}[proposition]{{\bf Corollary}}
\newtheorem{lemma}[proposition]{{\bf Lemma}}

% tên của mục lục và tài liệu tham khảo - mặc định
%\AtBeginDocument{\renewcommand*\contentsname{Contents}}
%\AtBeginDocument{\renewcommand*\refname{References}}

\AtBeginDocument{\renewcommand*\contentsname{Mục lục}}
\AtBeginDocument{\renewcommand*\refname{Tài liệu tham khảo}}


%\usepackage{fancyhdr}
\setlength{\headheight}{40pt}
\pagestyle{fancy}
\fancyhead{} % clear all header fields
\fancyhead[L]{
	\begin{tabular}{rl}
		\begin{picture}(25,15)(0,0)
			\put(0,-8){\includegraphics[width=8mm, height=8mm]{hcmut.png}}
			%\put(0,-8){\epsfig{width=10mm,figure=hcmut.eps}}
		\end{picture}&
		%\includegraphics[width=8mm, height=8mm]{hcmut.png} & %
		\begin{tabular}{l}
			\textbf{\bf \ttfamily Đại học Quốc gia Thành phố Hồ Chí Minh}\\
			\textbf{\bf \ttfamily Khoa khoa học và kỹ thuật máy tính}
		\end{tabular} 	
	\end{tabular}
}
\fancyhead[R]{
	\begin{tabular}{l}
		\tiny \bf \\
		\tiny \bf 
\end{tabular}  }
\fancyfoot{} % clear all footer fields
\fancyfoot[L]{\scriptsize \ttfamily Bài tập lớn Xử lý Ảnh số và Thị giác Máy tính - Học kỳ 2024 - 2025}
\fancyfoot[R]{\scriptsize \ttfamily Trang {\thepage}/\pageref{LastPage}}
\renewcommand{\headrulewidth}{0.3pt}
\renewcommand{\footrulewidth}{0.3pt}


%%%
\setcounter{secnumdepth}{4}
\setcounter{tocdepth}{3}
\makeatletter
\newcounter {subsubsubsection}[subsubsection]
\renewcommand\thesubsubsubsection{\thesubsubsection .\@alph\c@subsubsubsection}
\newcommand\subsubsubsection{\@startsection{subsubsubsection}{4}{\z@}%
	{-3.25ex\@plus -1ex \@minus -.2ex}%
	{1.5ex \@plus .2ex}%
	{\normalfont\normalsize\bfseries}}
\newcommand*\l@subsubsubsection{\@dottedtocline{3}{10.0em}{4.1em}}
\newcommand*{\subsubsubsectionmark}[1]{}
\makeatother

\begin{document}
	
	\begin{titlepage}
		\begin{center}
			ĐẠI HỌC QUỐC GIA THÀNH PHỐ HỒ CHÍ MINH \\
			TRƯỜNG ĐẠI HỌC BÁCH KHOA \\
			KHOA KHOA HỌC VÀ KỸ THUẬT MÁY TÍNH
		\end{center}
		
		\vspace{1cm}
		
		\begin{figure}[h!]
			\begin{center}
				\includegraphics[width=3cm]{hcmut.png}
			\end{center}
		\end{figure}
		
		\vspace{1cm}
		
		
		\begin{center}
			\begin{tabular}{c}
				
				\multicolumn{1}{l}{\textbf{{\Large XỬ LÝ ẢNH SỐ VÀ THỊ GIÁC MÁY TÍNH (CO3057)}}}\\
				\\
				\hline
				\\
				\multicolumn{1}{l}{\textbf{{\Large Bài tập lớn}}}\\
				\\
				\textbf{{\Huge Tiêu đề}}\\
				\\
				\hline
			\end{tabular}
		\end{center}
		
		\vspace{3cm}
		
		\begin{table}[h]
			\begin{tabular}{rrl}
				\hspace{5 cm} & Giảng viên hướng dẫn: & Nguyễn Đức Dũng\\
				& Sinh viên thực hiện: & Lê Quang Hiển - 2113376 \\
				& & Lê Trung Trực - MSSV \\
			\end{tabular}
		\end{table}
		
		\begin{center}
			{\footnotesize THÀNH PHỐ HỒ CHÍ MINH, THÁNG 12, NĂM 2024}
		\end{center}
	\end{titlepage}
	
	
	%\thispagestyle{empty}
	
	%%-----TABLE OF CONTENT ------------------
	\newpage
	%\renewcommand{\contentsname}{Mục lục}
	\tableofcontents
	\newpage
	%-----------------------------------------
	
	%%%%%%%%%%%%%%%%%%%%%%%%%%%%%%%%%
	\section*{Danh sách thành viên \& nhiệm vụ được phân công}
	\addcontentsline{toc}{section}{\protect\numberline{}Danh sách thành viên \& nhiệm vụ được phân công}
	\begin{flushleft}
		\begin{tabular}{|c|c|c|l|c|}
			\hline
			\multirow{3}{*}{\textbf{STT}} & \multirow{3}{*}{\textbf{Họ và tên}} & \multirow{3}{*}{\textbf{Mã số}} & \multirow{3}{*}{\textbf{Nhiệm vụ được phân công}} & \multirow{3}{*}{\textbf{Mức độ}}\\
			& & & &\\
			& & \textbf{sinh viên}& & \textbf{hoàn thành}\\
			
			\hline 
			%%%%%Student 1%%%%%%%%%%
			\multirow{5}{*}{1} & \multirow{5}{*}{Lê Quang Hiển} & \multirow{5}{*}{2113376} & -Phân chia công việc nhóm & \multirow{5}{*}{100\%}\\
			& & & -Kiểm tra, giám sát tiến độ  & \\
			& & & thực hiện bài tập lớn & \\
			& &  & -Thực hiện file báo cáo &\\
			& &  & -Nộp bài tập lớn &\\
			\hline 
			%%%%%Student 2%%%%%%%%%%%
			\multirow{3}{*}{2} & \multirow{3}{*}{Lê Đình Huy} & \multirow{3}{*}{2113481} & \multirow{3}{*}{-Thực hiện bài tập 3} & \multirow{3}{*}{100\%}\\
			& &  &  &\\
			& &  &  &\\
			\hline
		\end{tabular}
	\end{flushleft}
	
	\newpage
	%%%%%%%%%%%%%%%%%%%%%%%%%%%%%%%%%
\section{Relation \& Counting}
\subsection{Problem 1}
...

\subsection{Problem 2}
...

\subsection{Bonus exercises}
...

%%%%%%%%%%%%%%%%%%%%%%%%%%%%%%%%%
\section{Probabilty}
\subsection{Problem 1}
...

\subsection{Problem 2}
...

\subsection{Bonus exercises}
...

%%%%%%%%%%%%%%%%%%%%%%%%%%%%%%%%%


\section{Graph}
\subsection{Problem 1}
...

\subsection{Problem 2}
...

\subsection{Bonus exercises}
...

%%%%%%%%%%%%%%%%%%%%%%%%%%%%%%%%%%%%%%%%%%%%%%%%%%%%
\newpage
\addcontentsline{toc}{section}{\protect\numberline{}Tài liệu tham khảo}
%\renewcommand\refname{Tài liệu tham khảo}
\begin{thebibliography}{80}
	\bibitem{bib1}
	...
	\bibitem{bib2}
	...
	
	
\end{thebibliography}
\end{document}

