\documentclass[a4paper]{article}
\usepackage{a4wide,amssymb,epsfig,latexsym,multicol,array,hhline,fancyhdr}
\usepackage{vntex}
\usepackage{amsmath}
\usepackage{cases}
\usepackage{lastpage}
\usepackage[lined,boxed,commentsnumbered]{algorithm2e}
\usepackage{enumerate}
\usepackage{color}
\usepackage{graphicx}							% Standard graphics package
\usepackage{array}
\usepackage{tabularx, caption}
\usepackage{multirow}
\usepackage{multicol}
\usepackage{rotating}
\usepackage{graphics}
\usepackage{geometry}
\usepackage{setspace}
\usepackage{epsfig}
\usepackage{tikz}
\usepackage{amsfonts}
\usetikzlibrary{arrows,snakes,backgrounds}
\usepackage{hyperref}
%\usepackage{txfonts}
\usepackage{mathdots}
\usepackage{titlesec}
\usepackage{float}
\usepackage{listings}
\usepackage{xcolor}  % Tùy chọn để thay đổi màu sắc của mã


% Định nghĩa màu sắc từ theme Dracula
\definecolor{dracula-bg}{RGB}{40, 42, 54}         % Màu nền
\definecolor{dracula-current-line}{RGB}{68, 71, 90} % Màu dòng hiện tại
\definecolor{dracula-selection}{RGB}{68, 71, 90}    % Màu khi chọn
\definecolor{dracula-foreground}{RGB}{248, 248, 242} % Màu chữ
\definecolor{dracula-comment}{RGB}{98, 114, 164}   % Màu comment
\definecolor{dracula-cyan}{RGB}{139, 233, 253}     % Màu cyan (biến đặc biệt)
\definecolor{dracula-green}{RGB}{80, 250, 123}     % Màu xanh lá (hàm)
\definecolor{dracula-orange}{RGB}{255, 184, 108}   % Màu cam (tham số)
\definecolor{dracula-pink}{RGB}{255, 121, 198}     % Màu hồng (keyword)
\definecolor{dracula-purple}{RGB}{189, 147, 249}   % Màu tím (biến, class)
\definecolor{dracula-red}{RGB}{255, 85, 85}        % Màu đỏ (cảnh báo)
\definecolor{dracula-yellow}{RGB}{241, 250, 140}   % Màu vàng (chuỗi)

% Định nghĩa style cho Python
\lstset{frame=tb,
	language=Python,
	backgroundcolor=\color{dracula-bg},          % Màu nền
	basicstyle=\ttfamily\footnotesize\color{dracula-foreground}, % Font và màu chữ chính
	keywordstyle=\color{dracula-pink},           % Từ khóa
	commentstyle=\color{dracula-comment},        % Bình luận
	stringstyle=\color{dracula-yellow},          % Chuỗi
	numberstyle=\tiny\color{dracula-purple},     % Số dòng
	identifierstyle=\color{dracula-foreground},  % Tên biến
	tabsize=4,                                   % Kích thước tab
%	frame=single,                                % Khung xung quanh
	showspaces=false,                            % Không hiển thị khoảng trắng
	showstringspaces=false,                      % Không hiển thị khoảng trắng trong chuỗi
	showtabs=false,                              % Không hiển thị tab
	numbers=left,                                % Đánh số dòng bên trái
	breaklines=true,                             % Tự động xuống dòng
	breakatwhitespace=true,                      % Xuống dòng tại khoảng trắng
	captionpos=b,                                 % Vị trí chú thích ở dưới
	keywords={cv2, cv}
}


\newcommand{\applysectionstyle}{\titleformat{\section}[block]
	{\bfseries\huge}    % Định dạng cho section (đậm và kích thước chữ lớn)
	{Chương \thesection} % Thêm chữ "Chương" và số chương
	{1em}               % Khoảng cách giữa số chương và tên chương
	{}                  % Không có nội dung trước tiêu đề
}



\hypersetup{urlcolor=blue,linkcolor=black,citecolor=black,colorlinks=true} 
%\usepackage{pstcol} 								% PSTricks with the standard color package

\newtheorem{theorem}{{\bf Theorem}}
\newtheorem{property}{{\bf Property}}
\newtheorem{proposition}{{\bf Proposition}}
\newtheorem{corollary}[proposition]{{\bf Corollary}}
\newtheorem{lemma}[proposition]{{\bf Lemma}}

% tên của mục lục và tài liệu tham khảo - mặc định
%\AtBeginDocument{\renewcommand*\contentsname{Contents}}
%\AtBeginDocument{\renewcommand*\refname{References}}

\AtBeginDocument{\renewcommand*\contentsname{Mục lục}}
\AtBeginDocument{\renewcommand*\refname{Tài liệu tham khảo}}


%\usepackage{fancyhdr}
\setlength{\headheight}{40pt}
\pagestyle{fancy}
\fancyhead{} % clear all header fields
\fancyhead[L]{
	\begin{tabular}{rl}
		\begin{picture}(25,15)(0,0)
			\put(0,-8){\includegraphics[width=8mm, height=8mm]{hcmut.png}}
			%\put(0,-8){\epsfig{width=10mm,figure=hcmut.eps}}
		\end{picture}&
		%\includegraphics[width=8mm, height=8mm]{hcmut.png} & %
		\begin{tabular}{l}
			\textbf{\bf \ttfamily Đại học Quốc gia Thành phố Hồ Chí Minh}\\
			\textbf{\bf \ttfamily Khoa khoa học và kỹ thuật máy tính}
		\end{tabular} 	
	\end{tabular}
}
\fancyhead[R]{
	\begin{tabular}{l}
		\tiny \bf \\
		\tiny \bf 
\end{tabular}  }
\fancyfoot{} % clear all footer fields
\fancyfoot[L]{\scriptsize \ttfamily Bài tập lớn Xử lý Ảnh số và Thị giác Máy tính - Học kỳ 2024 - 2025}
\fancyfoot[R]{\scriptsize \ttfamily Trang {\thepage}/\pageref{LastPage}}
\renewcommand{\headrulewidth}{0.3pt}
\renewcommand{\footrulewidth}{0.3pt}


%%%
\setcounter{secnumdepth}{4}
\setcounter{tocdepth}{3}
\makeatletter
\newcounter {subsubsubsection}[subsubsection]
\renewcommand\thesubsubsubsection{\thesubsubsection .\@alph\c@subsubsubsection}
\newcommand\subsubsubsection{\@startsection{subsubsubsection}{4}{\z@}%
	{-3.25ex\@plus -1ex \@minus -.2ex}%
	{1.5ex \@plus .2ex}%
	{\normalfont\normalsize\bfseries}}
\newcommand*\l@subsubsubsection{\@dottedtocline{3}{10.0em}{4.1em}}
\newcommand*{\subsubsubsectionmark}[1]{}
\makeatother

\begin{document}
	
	\begin{titlepage}
		\begin{center}
			ĐẠI HỌC QUỐC GIA THÀNH PHỐ HỒ CHÍ MINH \\
			TRƯỜNG ĐẠI HỌC BÁCH KHOA \\
			KHOA KHOA HỌC VÀ KỸ THUẬT MÁY TÍNH
		\end{center}
		
		\vspace{1cm}
		
		\begin{figure}[h!]
			\begin{center}
				\includegraphics[width=3cm]{hcmut.png}
			\end{center}
		\end{figure}
		
		\vspace{1cm}
		
		
		\begin{center}
			\begin{tabular}{c}
				
				\multicolumn{1}{l}{\textbf{{\Large XỬ LÝ ẢNH SỐ VÀ THỊ GIÁC MÁY TÍNH (CO3057)}}}\\
				\\
				\hline
				\\
				\multicolumn{1}{l}{\textbf{{\Large Bài tập lớn}}}\\
				\\
				\textbf{{\Huge Khớp đặc trưng giữa các hình ảnh}}\\
				\\
				\hline
			\end{tabular}
		\end{center}
		
		\vspace{3cm}
		
		\begin{table}[h]
			\begin{tabular}{rrl}
				\hspace{5 cm} & Giảng viên hướng dẫn: & Nguyễn Đức Dũng\\
				& Sinh viên thực hiện: & Lê Quang Hiển - 2113376 \\
				& & Lê Trung Trực - 2115161 \\
			\end{tabular}
		\end{table}
		
		\begin{center}
			{\footnotesize THÀNH PHỐ HỒ CHÍ MINH, THÁNG 12, NĂM 2024}
		\end{center}
	\end{titlepage}
	
	
	%\thispagestyle{empty}
	
	%%-----TABLE OF CONTENT ------------------
	\newpage
	%\renewcommand{\contentsname}{Mục lục}
	\tableofcontents
	\newpage
	%-----------------------------------------
	
	%%%%%%%%%%%%%%%%%%%%%%%%%%%%%%%%%
	\section*{Danh sách thành viên \& nhiệm vụ được phân công}
	\addcontentsline{toc}{section}{\protect\numberline{}Danh sách thành viên \& nhiệm vụ được phân công}
	\begin{flushleft}
		\begin{tabular}{|c|c|c|l|c|}
			\hline
			\multirow{3}{*}{\textbf{STT}} & \multirow{3}{*}{\textbf{Họ và tên}} & \multirow{3}{*}{\textbf{Mã số}} & \multirow{3}{*}{\textbf{Nhiệm vụ được phân công}} & \multirow{3}{*}{\textbf{Mức độ}}\\
			& & & &\\
			& & \textbf{sinh viên}& & \textbf{hoàn thành}\\
			
			\hline 
			%%%%%Student 1%%%%%%%%%%
			\multirow{5}{*}{1} & \multirow{5}{*}{Lê Quang Hiển} & \multirow{5}{*}{2113376} & -Phân chia công việc nhóm & \multirow{5}{*}{100\%}\\
			& & & -Kiểm tra, giám sát tiến độ  & \\
			& & & thực hiện bài tập lớn & \\
			& &  & -Thực hiện file báo cáo &\\
			& &  & -Nộp bài tập lớn &\\
			\hline 
			%%%%%Student 2%%%%%%%%%%%
			\multirow{3}{*}{2} & \multirow{3}{*}{Lê Trung Trực} & \multirow{3}{*}{2115161} & \multirow{3}{*}{-Thực hiện bài tập 3} & \multirow{3}{*}{100\%}\\
			& &  &  &\\
			& &  &  &\\
			\hline
		\end{tabular}
	\end{flushleft}

%%%%%%%%%%%%%%%%%%%%%%%%%%%%%%%%%%%%%%%%%%%%%%%%%%%%
\newpage
\applysectionstyle
\newpage
\section{Giới Thiệu}
\subsection{Tổng quan về bài toán "Feature Matching"}
Feature mathcing là một kỹ thuật quan trọng trong thị giác máy tính. Feature matching hay đối sánh đặc trưng ảnh giúp máy tính có khả năng nhìn và nhận dạng được đối tượng thông qua đầu vào là ảnh hoặc video. Ý tưởng của thuật toán đối sánh đặc trưng là so sánh các điểm tương quan giữa hai ảnh về cùng một vật thể hoặc bối cảnh. Để nhận dạng được một ảnh thì một hệ thống thị giác máy tính bao gồm 4 thành phần chính được sử dụng. 4 thành phần chính này là:

\begin{enumerate}
	\item Dò tìm điểm khóa (keypoint detection)
	\item Mô tả điểm khóa (keypoint description)
	\item Lập chỉ mục các bộ mô tả (descriptor indexing)
	\item Đối sánh (matching)
\end{enumerate}

Trong đó hai thành phần giữ vai trò quan trọng nhất là mô tả điểm khóa và lập chỉ mục vì chúng quyết định đến độ chính xác và thời gian xử lý của quá trình đối sánh.\\

Feature matching có ý nghĩa rất lớn trong thực tế. Các ứng dụng có thể được kể đến là image alignment, phục dựng bối cảnh 3D từ ảnh 2D, theo dõi chuyển động, nhận diện vật thể...

\subsection{Các vấn đề chính trong việc khớp đặc trưng}
Trước khi tiến hành khớp đặc trưng ảnh, ảnh phải trải qua một bước trích chọn đặc trưng ảnh. Việc trích chọn được coi là thành công khi các đặc trưng thỏa mãn các tính chất bất biến sau: Bất biến với phép tỷ lệ (sacle invariance), bất biến với phép xoay (rotation invariance), bất biến với cường độ sáng (intensity invariance). Dù ở trong điều kiện nào, kết quả trả về trong các trường hợp vẫn phải là một đặc trưng nhất quán của đối tượng đó. Bên cạnh đó, phép trích chọn đặc trưng cũng cần thỏa mãn những tính chất như: tính độc nhất và đại diện (distinctiveness); tính chính xác về vị trí, hình dạng và tỷ lệ; tính hiệu quả; bền vững khi ảnh bị nhiễu, nhòe, nén hoặc biến đổi về hình dạng. Khi đã đảm bảo được tính ổn định và chính xác của các đặc trưng, bước đối sánh đặc trưng mới có thể được tiến hành hiệu quả.\\

Thuật toán đối sánh đặc trưng đơn giản nhất là tìm kiếm vét cạn cơ sở dữ liệu chứa tập các bộ mô tả mẫu. Vấn đề của cách tiếp cận này là nếu số lượng bộ mô tả mẫu lớn và không gian có nhiều chiều thì độ phức tạp tính toán cao, tìm kiếm lâu. Do đó, giải thuật này không đáp ứng được yêu cầu về thời gian trong các ứng dụng thời gian thực.\\

Để khắc phục nhược điểm trên, kỹ thuật đối sánh gần đúng sẽ được sử dụng để tối ưu về mặt thời gian. Kỹ thuật này chú trọng lập chỉ mục (descriptor indexing) dựa trên bộ mô tả đã được xây dựng. Từ đó, việc đối sánh ảnh chỉ được thực hiện trên một phần nhỏ của các bản mẫu tiềm năng mà không nhất thiết phải vét cạn cơ sở dữ liệu. Trong trường hợp lý tưởng, kết quả của kỹ thuật này là một tập các đối sánh sao cho mỗi đối sánh là một ánh xạ một-một giữa một đặc trưng của ảnh đầu vào và một đặc trưng tương ứng trong ảnh mẫu. Tuy nhiên trong thực tế, kết quả thu được thường là những ánh xạ một-nhiều hoặc nhiều-nhiều. Bên cạnh đó, kỹ thuật này còn có những hạn chế khác như hiệu năng tìm kiềm chính xác có thể không cao, yêu cầu nhiều không gian bộ nhớ, xây dựng cấu trúc chỉ mục phức tạp, độ phức tạp tính toán cao... Do đó, một bước hậu xử lý các đối sánh là cần thiết để loại bỏ những đối sánh sai.

\subsection{Mục tiêu nghiên cứu}
Mục tiêu nghiên cứu của đề tài này là hiểu được quy trình để nhận dạng các vật thể cần quan tâm trong ảnh và xác định được vật thể đó có trùng khớp với các mẫu có sẵn dù vật thể đang xét được thu với nhiều điều kiện khác nhau như xoay, thay đổi kích cỡ, thay đổi vị trí, góc nhìn hoặc điều kiện sáng. Thêm vào đó, sinh viên có cơ hội thực hành và hiểu hơn nữa tầm quan trọng của feature matching trong các ứng dụng thực tế của lĩnh vực xử lý ảnh số và thị giác máy tính.
\newpage
\section{Cơ Sở Lý Thuyết}
\subsection{Trích chọn biên ảnh}
Biên ảnh là điểm mà tại đó có sự thay đổi đột ngột về giá trị điểm ảnh só với các điểm lân cận. Xác định được biên ảnh là tiền đề quan trọng để xác định hình dáng và thông tin tổng thể của một đối tượng trong ảnh.\\

Hiệu quả của kỹ thuật dò biên được đánh giá qua các tiêu chí:
\begin{itemize}
	\item Bền vững với nhiều (robust to noise): Lọc ra được các điểm biên và các điểm nhiễu vì các điểm này đều có tần số cao.
	\item Phát hiện chính xác vị trí điểm biên (good localization)
	\item Phát hiện đúng điểm biên (single response): những điểm không phải biên thì không được ghi nhận vào kết quả trả về.
\end{itemize}
 
Sau đây là những kỹ thuật dò biên cơ bản và phổ biến
\subsubsection{Kỹ thuật gradient}
Gradient là một đại lượng biểu thị cho sự thay đổi của một giá trị điểm ảnh theo một hướng nào đó. Trong kỹ thuật gradient, vector gradient được dùng để xác định biên ảnh. Vector gradient được thể hiện bằng hai độ đo là:
\begin{itemize}
	\item Biên độ gradient: mức độ thay đổi của giá trị điểm ảnh.
	\item Hướng gradient: hướng của vector pháp tuyến vuông góc với đường biên tại điểm ảnh đang xét.
\end{itemize}
Muốn xác định được đường biên thì phải xác định được các điểm mà tại đó biên độ gradient đạt cực đại cục bộ (hoặc lớn hơn một ngưỡng nào đó). Tiếp theo, vector chỉ phương của đường biên có thể được tính toán từ hướng của gradient cộng cho $ 90^\circ $.

\subsubsection{Kỹ thuật Canny}

Kỹ thuật canny là một trong những kỹ thuật dó biên tốt. Nó đáp ứng đủ các tiêu chí đánh giá của giải thuật và cho ra kết quả chính xác hơn kỹ thuật gradient. Kỹ thuật canny bao gồm nhiều bước:
\begin{enumerate}
	\item Lọc nhiễu: Giảm bớt sự ảnh hưởng của nhiễu, làm mượt ảnh bằng cách sử dụng bộ lọc thông thấp Gaussian.
	\item Tìm gradients: Tương tự như kỹ thuật gradient, thuật toán sử dụng mặt nạ Sobel để tính vector gradient tại mỗi điểm ảnh.
	\item Loại các điểm không cực đại (non-maximum suppresion): Làm mỏng đường biên bằng cách loại bỏ những điểm không cực đại.
	\item Phân ngưỡng kép (double threshholding): Không giống như kỹ thuật dựa trên gradient chỉ dùng một giá trị ngưỡng để để xác định điểm biên, kỹ thuật Canny dùng ngưỡng kép bao gồm một ngưỡng cao và một ngưỡng thấp.
\end{enumerate}
\subsection{Các kỹ thuật dò tìm điểm khóa phổ biến}
\subsubsection{Bộ dò tìm điểm góc (Corner detection)}
Điểm góc là những điểm giao của ít nhất hai đường biên. Một thuật toán dò tìm điểm góc phổ biến là Haris Corner detector.\\

Nguyên lý của bộ dò tìm này là dùng một của sổ có kích thước nhỏ để quét qua mọi vị trí trên ảnh nhằm phân lại các vị trí đó thành:
\begin{itemize}
	\item Vùng đồng nhất: điểm ảnh trong vùng này khá giống với các điểm ảnh lân cận theo mọi hướng.
	\item Vị trí biên ảnh: các điểm trên vị trí này khá giống với các điểm lân cận trong cửa sổ theo hướng của đường biên.
	\item Vị trí điểm góc: Các điểm góc là những điểm rất khác so với các điểm ảnh lân cận theo mọi hướng.
\end{itemize}
\subsubsection{Bộ dò tìm blob}
Blob là một đặc trưng chỉ một vùng ảnh mà các điểm ảnh trong vùng tương đồng nhau về mức xám hoặc màu. Bản chất của việc dò tìm blob là tìm những điểm đặc trưng (interest point) sao cho những điểm này là cục trị địa phương của một hàm số nào đó. Có hai hàm phổ biến để tìm blob là LoG (Lapalce of Gaussian) và DoG (Difference of Gaussian).

\subsection{Các phương pháp mô tả đặc trưng phổ biến}
Sau khi các định được các điểm khóa, các điểm này cần phải được đặc tả để xác định tính đại diện của chúng. Bộ mô tả đặc trưng là một biễu diễn số học của một vùng trong ảnh tương ứng với một điểm đặc trưng. Nhờ có bộ mô tả thì ta mới có cơ sở để đối sánh các điểm khóa.

\subsubsection{Scale-invariant feature transform (SIFT)}
Mỗi điểm khóa sẽ được đặc tả bằng một vector số thực 128 chứa thông tin về biên độ và hướng của gradient của các điểm lân cận điểm khóa. Đúng như cái tên của nó SIFT cho ra kết quả bền vùng với phép tỷ lệ. Không chỉ có thế, bộ mô tả này còn bền vững với các phép biến đổi ảnh thường thấy như phép xoay, dịch chuyển hay sự tăng giảm độ sáng ảnh.

\subsubsection{Speeded up robust features (SURF)}
Bộ mô tả SURF là một thay thế nhằm giữ vững tính chính xác cao như SIFT những cải thiện đáng kể về hiệu năng tính toán. Bộ mô tả SURF được xây dựng bằng cách thống kê những đặc trưng về histogram của ảnh dựa trên các đáp ứng của hàm wavelet dạng Haar. Kích thước của bộ mô tả SURF là một vector 64 chiều bao gồm 16 histogram, mỗi histogram 4 chiều.

\subsubsection{Binary robust independent elementary features (BRIEF)}

BRIEF biễu diễn một điểm khóa bằng mỗi chuỗi các giá trị nhị phân. Các giá trị nhị phân này là kết quả của sự so sánh mức xám tại hai điểm bất kỳ trong một tập thuộc ảnh. Mỗi tập này có thể được sinh bằng 5 cách cấu hình. BRIEFT hiệu quả hơn hẳn 2 bộ mô tả trước đó trong tính toán thời gian thực với độ phức tạp tính toán thấp hơn.

\subsection{Các phương pháp đối sánh đặc trưng phổ biến}
Sau khi đã trích chọn được tập các vector đặc trưng (feature vector) hoặc bộ mô tả (descriptors) về đối tượng của ảnh, các bộ mô tả này sẽ được đem đi đối chiếu với một cơ sử dữ liệu chứa các bản mẫu.

\subsubsection{Đối sánh sử dụng brute-force}
Đối sánh dựa trên tìm kiếm vét cạn có thể được phát biểu bằng một bài toán:
\begin{itemize}
	\item $X$: cơ sở dữ liệu các vector đặc trưng.
	\item $q$: vector truy vấn.
\end{itemize}
Bài toán được giải quyết bằng cách duyệt qua mọi phần tử $p \in X$ và tính khoảng cách 
\[ h = dist(p, q) \]
 Khi có khoảng cách nhỏ hơn thì cập nhật lại h với giá trị nhỏ nhất (tối ưu nhất).
 
\subsubsection{Đối sánh dựa trên lập chỉ mục}
Thông thường, đối sánh dựa trên lập chỉ mục là một lựa chọn tối ưu hơn tìm kiếm vét cạn vì tối ưu về mặt thời gian. Trong cách tiếp cận này, cơ sở dữ liệu sẽ được tổ chức và sắp xếp lại sao cho quá trình duyệt chỉ thực hiện trên một phần nhỏ các bản mẫu mà vẫn cho ra kết quả tốt nhất. Kỹ thuật này có thể được chia thành 3 phương pháp:
\begin{itemize}
	\item Dựa trên phân cụm (clustering-based approach): lập chỉ mục bằng một cấu trúc dữ liệu cây phân cấp. Ở mỗi cấp của cây, một thuật toán phân cụm được dùng để chia tập dữ liệu ở một nút thành nhiều cụm con. Quá trình này được lặp với mỗi cụm con cho đến khi số lượng đặc trưng của mỗi cụm con đủ nhỏ. Khi đối sánh, khoảng cách h sẽ được tính từ vector truy vấn đến mỗi nút con của nút gốc. Nút con có kết quả nhỏ nhất sẽ được chọn để tiếp tục duyệt theo chiều sâu cho đến khi tìm đến một nút lá. Một thuật toán phân cụm phổ biến là Kd-Tree thuộc họ Nearest neighbor search.
	\item Dựa trên phân hoạch không gian (space-based approach): tương đối giống lập chỉ mục bằng cây phân cấp nhưng cách tiếp cận này sử dụng phương pháp phân hoạch không gian để chia nhỏ tập các vector đặc trưng. Cấu trúc dữ liệu được tạo ra là cây phân hoạch (cây cân bằng). Thuật toán phân hoạch phổ biến là cây tìm kiếm KD-tree, tức là tập ban đầu sẽ được sắp xếp và chia 2 dựa trên điểm trung vị.
	\item Dựa trên hàm băm (hashing-based approach): phương pháp lập chỉ mục này chú trọng xây dựng một họ các hàm băm sao cho xác suất các điểm gần nhau được băm vào cùng một ô tỷ lệ thuận với sự giống nhau của các điểm đó.
\end{itemize}
\subsection{Các kỹ thuật hiệu chỉnh sau đối sánh}
Hiệu chỉnh sau đối sánh là bước hậu xử lý để loại ra những đối sánh sai giữa ảnh đầu vào và ảnh mẫu trong cơ sở dữ liệu. 

\subsubsection{Kỹ thuật RANSAC}
RANSAC được dùng để hậu xử lý cái sai số của quá trình đối sánh bằng cách kiểm tra tính nhất quán hình học. Tính nhất quán hình học này được biểu diễn bằng một mô hình biến đổi (phép biến đối). Để xây dựng mô hình biến đổi, ta cần xác định các tham số của nó từ tập dữ liệu quan sát ban đầu bao gồm cả dữ liệu sạch (inliners) và dữ liệu nhiễu (outliers). Dữ liệu sạch là những điểm dữ liệu thỏa mãn mô hình biến đổi cần xây dựng và ngược lại, dữ liệu nhiều sẽ không thỏa mãn mô hình. Mức độ hiệu quả của RANSAC chỉ được tận dụng khi số lượng các điểm dữ liệu sách chiếm hơn phân nửa.

\subsubsection{Kỹ thuật biến đổi Hough}
Kỹ thuật biến đổi Hough cũng có phần tương đồng với kỹ thuật RANSAC. Nhưng thay vì chỉ tìm 1 mô hình biến đổi, kỹ thuật này sẽ tìm nhiều mô hình tham số tiềm năng. Thay vì ước lượng các tham sô ngẫu nhiên, kỹ thuật biến đổi Hough tiếp cận bằng phương pháp bỏ phiếu (voting) bằng một bảng băm đa chiều với các khóa là tham số của mô hình biến đổi. Giá trị của hàm băm là số lượng các đối sánh nhất quán tương ứng với tham số của phép biến đổi. Các mô hình biến đổi được cho là tiềm năng khi có số lượng được "bỏ phiếu" ít nhất là 3.
\newpage
\section{Phương Pháp Nghiên Cứu}
\subsection{Quy trình khớp đặc trưng giữa hai hình ảnh}
\subsection{Công cụ/Thư viện sử dụng}
\newpage
\section{Đánh Giá Hiệu Quả Của Các Thuật Toán}

Hiệu quả của các thuật toán phát hiện đặc trưng như SIFT, SURF, và ORB đóng vai trò quan trọng trong xử lý ảnh và thị giác máy tính. Việc đánh giá hiệu suất của các thuật toán này dựa trên các tiêu chí như tốc độ xử lý, độ chính xác, và khả năng ứng phó với các điều kiện biến đổi như góc quay, thu phóng hoặc thay đổi ánh sáng là rất cần thiết. Mỗi thuật toán mang lại những lợi thế riêng trong việc phát hiện và mô tả đặc trưng, góp phần nâng cao chất lượng khớp đặc trưng trong nhiều ứng dụng thực tế. Nhóm nghiên cứu xin được tham khảo bài viết \textit{"A Comparison of SIFT, SURF and ORB on OpenCV"} của \textit{Mikhail Kennerley}\cite{eval-feature}.
\subsection{Phương pháp đánh giá}
\subsubsection{Tốc độ xử lý}
\begin{figure}[H]
	\centering
	\includegraphics[width=12cm]{images/SpeedComparison.png}
	\caption{Tốc độ tính toán của bộ mô tả điểm khóa}
\end{figure}
Có thể thấy tốc độ xử lý của bộ mô tả SURF đã cải thiện so với SIFT vì sử dụng các phép tính đơn giản hơn rất nhiều. Tuy nhiên đó cũng chỉ là khoảng cách tương đối nhỏ. Trong khi đó, bộ mô tả ORB có thời gian tính toán nhanh vượt trội so với hai phương pháp trên.
\subsubsection{Độ chính xác}
\begin{figure}[H]
	\centering
	\includegraphics[width=12cm]{images/AccuracyComparison.png}
	\caption{Số lượng các điểm đặc trưng tìm được}
\end{figure}
Có thể thấy bộ mô tả ORB thu được số lượng điểm đặc trưng nhiều nhất, gần gấp ba lần so với SURF. SIFT thì cho ra số lượng điểm đặc trưng thấp nhất. Lý do cho điều này là ORB không tìm những điểm cực trị cục bộ một cách tuyệt đối trong lân cận mà chỉ cần là cực trị của tập hợp các điểm liên tiếp mà ORB sử dụng để so sánh và xác định cực đại cục bộ.
\subsubsection{Độ tin cậy}
Trong cả hai trường hợp, độ tin cậy của bộ mô tả ORB luôn cao nhất, có lúc đạt đến tuyệt đối.
\begin{figure}[H]
	\centering
	\includegraphics[width=12cm]{images/Precision.png}
	\caption{Tỷ lệ đối sánh đúng trong khi cường đồ sáng cao}
\end{figure}
Trong trường hợp ảnh được tăng cường độ sáng, độ tin cậy của bộ mô tả SURF và SIFT là bằng nhau.
\begin{figure}[H]
	\centering
	\includegraphics[width=12cm]{images/Precision1.png}
	\caption{Tỷ lệ đối sánh đúng khi ảnh bị xoay}
\end{figure}
Còn trong trường hợp ảnh được biến đổi xoay, độ tin cậy của bộ mô tả tăng lên 100\% trong khi độ tin cậy của SIFT là hầu như không đổi.

\subsection{Kết quả và so sánh}
Hiệu quả của bộ mô tả ORB được chứng minh là hiệu quả nhất và kết quả đầu ra bất biến với nhiều phép biến đổi.
\newpage
\section{Kết Luận}

Việc khớp đặc trưng giữa các hình ảnh là một bài toán quan trọng trong lĩnh vực xử lý ảnh số và thị giác máy tính. Trong bài tiểu luận này, nhóm nghiên cứu đã tìm hiểu và áp dụng các kỹ thuật khớp đặc trưng như SIFT, ORB, FLANN, cùng với các kỹ thuật lọc như Lowe’s Ratio Test và RANSAC để nâng cao độ chính xác của các điểm khớp. Nội dung nghiên cứu đã tổng hợp các nguyên lý lý thuyết và triển khai qua thực nghiệm, từ đó mang lại những kết quả đáng khích lệ. Các phương pháp này không chỉ giúp cải thiện hiệu quả xử lý mà còn cung cấp những công cụ mạnh mẽ để giải quyết các vấn đề thực tế liên quan đến so khớp hình ảnh.\\

Trong quá trình triển khai, các thuật toán phát hiện đặc trưng như SIFT đã chứng minh được độ chính xác và độ ổn định cao, đặc biệt trong việc xử lý những hình ảnh có nhiều chi tiết phức tạp. SURF với tốc độ xử lý nhanh hơn SIFT cũng là một lựa chọn hiệu quả trong các bài toán yêu cầu thời gian xử lý ngắn hơn, nhưng độ chính xác có thể giảm trong trường hợp chi tiết phức tạp hoặc biến đổi hình ảnh lớn. ORB, là một thuật toán nhanh và tiết kiệm tài nguyên, phù hợp với các ứng dụng thời gian thực hoặc các hệ thống có hạn chế về tài nguyên, mặc dù độ chính xác của ORB thường thấp hơn so với SIFT và SURF trong các bài toán yêu cầu độ chi tiết cao.\\

FLANN với khả năng khớp nhanh chóng đã giúp đơn giản hóa quá trình tìm đối tượng khớp trong tập dữ liệu lớn. Để nâng cao độ chính xác, Lowe's Ratio Test đã loại bỏ nhiều kết quả sai, trong khi RANSAC giúp lọc những khớp đặc trưng bị ảnh hưởng bởi nhiễu tạp chất hoặc đối tượng ngoại lai. Các kết quả hình ảnh thực nghiệm đã minh chứng cho sự hợp lý của những kỹ thuật này. Thêm vào đó, việc kết hợp các phương pháp này trong một quy trình thống nhất đã tạo ra một hệ thống mạnh mẽ, hiệu quả và đáng tin cậy trong việc giải quyết bài toán khớp đặc trưng.\\

Tuy nhiên, trong quá trình nghiên cứu, một số hạn chế cũng được ghi nhận. Các thuật toán như SIFT dù có độ chính xác cao nhưng thời gian xử lý lại tương đối lâu, đặc biệt khi làm việc với tập dữ liệu lớn. SURF, dù nhanh hơn, vẫn gặp khó khăn trong các trường hợp hình ảnh bị biến dạng lớn hoặc góc chụp khác biệt (dựa theo bài báo so sánh). ORB tuy nhanh nhưng thường không đủ chính xác trong các bài toán phức tạp. FLANN có độ nhanh nhưng kết quả có thể bị sai khi dữ liệu đặc trưng không đủ phong phú. Hơn nữa, đối với các hình ảnh bị biến dạng quá lớn hoặc góc chụp quá khác biệt, kết quả không đạt được mức mong muốn. Những hạn chế này chỉ ra rằng việc kết hợp thêm các phương pháp mới hoặc cải tiến các thuật toán hiện tại là cần thiết để đạt được hiệu suất cao hơn.\\

Hướng nghiên cứu trong tương lai có thể tập trung vào việc áp dụng các thuật toán tiên tiến hơn như SuperPoint hay các mô hình học sâu cho việc khớp đặc trưng. Việc tối ưu hóa hiệu suất thông qua tối ưu phần cứng và đánh giá kỹ lưỡng trải nghiệm cũng là những hướng đi rõ rệt. Hơn nữa, tích hợp các thuật toán này vào những hệ thống thực tế như nhận diện khuôn mặt hay tính năng định vị đồng thời là bước nhảy vọt quan trọng. Các nghiên cứu này không chỉ cải thiện độ chính xác mà còn mở rộng khả năng ứng dụng của các kỹ thuật khớp đặc trưng vào nhiều lĩnh vực mới, đặc biệt trong các ngành công nghiệp sử dụng công nghệ thị giác máy tính.\\

Tổng quát lại, nghiên cứu về việc khớp đặc trưng giữa các hình ảnh đã minh chứng cho tầm quan trọng và tiềm năng lớn của lĩnh vực này trong xử lý ảnh và thị giác máy tính. Dù còn những hạn chế, đòi hỏi các nghiên cứu chuyên sâu để tối ưu cũng như tìm ra giải pháp tốt hơn, nhưng những kết quả đạt được là bước đệm quan trọng trong việc phát triển các ứng dụng thực tế. Sự kết hợp giữa lý thuyết và thực nghiệm trong nghiên cứu này đã tạo ra một nền tảng vững chắc cho nhóm nghiên cứu phát triển công trình tiếp theo trong tương lai.\\

Toàn bộ mã nguồn triển khai của các phương pháp được thảo luận trong bài viết có thể tham khảo tại kho lưu trữ GitHub của nhóm nghiên cứu: \url{https://github.com/hienlq16103/digital-image-processing-and-computer-vision}.
\newpage
\addcontentsline{toc}{section}{\protect\numberline{}Tài liệu tham khảo}
%\renewcommand\refname{Tài liệu tham khảo}
\begin{thebibliography}{80}
	\bibitem{bib1}
	...
	\bibitem{bib2}
	...
	
	
\end{thebibliography}

%%%%%%%%%%%%%%%%%%%%%%%%%%%%%%%%%%%%%%%%%%%%%%%%%%%%
\end{document}

